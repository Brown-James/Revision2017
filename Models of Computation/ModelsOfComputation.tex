\documentclass[11pt]{article}
\usepackage{imakeidx}
\usepackage{geometry}
\geometry{a4paper,
total={170mm, 257mm},
left = 30mm,
right = 30mm,
bottom = 30mm,
top = 30mm
}
\title{Models of Computation \linebreak Revision Notes}
\author{James Brown}
\makeindex
\begin{document}
	\pagenumbering{gobble}
	\maketitle
	\newpage
	\tableofcontents
	\newpage
	\pagenumbering{arabic}
	
	\section{Introduction}
	These are notes I have written in preparation for the upcoming 2017 Models of Computation exam. This year the module was run by Paul Levy (P.B.Levy@cs.bham.ac.uk).
	\linebreak
	This module is about problems and \textit{computers}. We ask ourselves:
	\begin{itemize}
		\item What problems can be solved on a computer?
		\item What problems can be solved on a computer with finitely many states?
		\item What problems can be solved on a computer with only finitely many states, but also a stack of unlimited size?
		\item What problems can be solved on a computer with only finitely many states, but also a tape of unlimited size that it can read and write to?
		\item What problems can be solved \textit{fast} on a computer?
		\item What does "fast" mean anyway?
		\item What does \textit{computer} mean anyway?
	\end{itemize}
	
	\section{Language Membership Problems and Regular Expressions}
	\subsection{Language Membership Problems}
	Suppose we have a set of characters $\Sigma$, which we will call the \textit{alphabet}\index{alphabet}.
	
	\subsection{Regex}
	\textbf{Regular Expressions}\index{Regular Expressions} are a useful notation for describing languages.
	
	
	\section{Finite State Automata}
	
	\section{Bisimulation and Minimisation}
	
	\section{The Halting Problem}
	
	\section{Properties of Code}
	
	\section{Turing Machines}
	
	\section{Church's Thesis}
	
	\section{Complexity and P}
	
	\section{NP}
	
	\section{Lambda-calculus}
	
\end{document}