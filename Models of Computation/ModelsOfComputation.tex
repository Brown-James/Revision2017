\documentclass[11pt]{article}
\usepackage{imakeidx}
\usepackage{geometry}
\geometry{a4paper,
total={170mm, 257mm},
left = 30mm,
right = 30mm,
bottom = 30mm,
top = 30mm
}
\title{Models of Computation \linebreak Revision Notes}
\author{James Brown}
\makeindex
\begin{document}
	\pagenumbering{gobble}
	\maketitle
	\newpage
	\tableofcontents
	\newpage
	\pagenumbering{arabic}
	
	\section{Introduction}
	These are notes I have written in preparation for the upcoming 2017 Models of Computation exam. This year the module was run by Paul Levy (P.B.Levy@cs.bham.ac.uk).
	\linebreak
	This module is about problems and \textit{computers}. We ask ourselves:
	\begin{itemize}
		\item What problems can be solved on a computer?
		\item What problems can be solved on a computer with finitely many states?
		\item What problems can be solved on a computer with only finitely many states, but also a stack of unlimited size?
		\item What problems can be solved on a computer with only finitely many states, but also a tape of unlimited size that it can read and write to?
		\item What problems can be solved \textit{fast} on a computer?
		\item What does "fast" mean anyway?
		\item What does \textit{computer} mean anyway?
	\end{itemize}
	
	\section{Language Membership Problems and Regular Expressions}
	\subsection{Language Membership Problems}
	Suppose we have a set of characters $\Sigma$, which we will call the \textit{alphabet}\index{alphabet}. A \textit{word}\index{word} is a finite sequence of characters, and we write $\Sigma^{*}$ for the set of all words. We can \textit{concatenate}\index{concatenate} words. A \textit{language}\index{language} is a set of words and a subset of $\Sigma^{*}$. Given a word, we want to know is it in the language or not? If we take an example alphabet ${a, b, c}$, here are some languages:
	\begin{itemize}
		\item All words which contain exactly 3 \textit{b}'s
		\item All words whose length is prime
		\item All words that have more \textit{b}'s than \textit{a}'s
		\item The words \textit{abc}, \textit{bac} and \textit{cb}
		\item No words at all
		\item The empty word
	\end{itemize}
	
	These examples are largely pretty useless, but this problem does have real world applications such as 
	\begin{itemize}
		\item Java has rules about what you can call a variable. Is the word read by the compiler a valid variable name?
		\item A user makes an account and enters a password, is it valid?
		\item A student has submitted code for an assignment, is it correct?
		\item Will this code crash when it's run?
	\end{itemize}
	
	In each one of these examples, we are provided with a word and we want to know whether it is an acceptable word. We want to make a computer tell us the answer.
	
	
	\subsection{Regex}
	\textbf{Regular Expressions}\index{Regular Expressions} are a useful notation for describing languages.
	
	
	\section{Finite State Automata}
	
	\section{Bisimulation and Minimisation}
	
	\section{The Halting Problem}
	
	\section{Properties of Code}
	
	\section{Turing Machines}
	
	\section{Church's Thesis}
	
	\section{Complexity and P}
	
	\section{NP}
	
	\section{Lambda-calculus}
	
\end{document}