\documentclass{article}
\usepackage{imakeidx}
\title{Computational Vision \linebreak Revision Notes}
\author{James Brown}
\makeindex
\begin{document}
	\pagenumbering{gobble}
	\maketitle
	\newpage
	\tableofcontents
	\newpage
	\pagenumbering{arabic}

	\section{Introduction}
	These are notes I have written in preparation of the 2017 Computation Vision exam. This year the module was run by Hamid Deghani (H.Dehghani@cs.bham.ac.uk).
	\linebreak \linebreak
	Computational vision is the acquisition of knowledge about objects and events in the environment through information processing of light emitted or reflected from objects. In short - we want to make a computer know what is where, by looking through information. We can also use computational vision to do automatic inference of properties of the world from images.

	\section{Human Vision}
	\par
	As humans we have evolved eyes which percieve the visible section of the electromagnetic spectrum, which falls between the wavelengths of 380nm - 760nm. Red light lies at the longer end (760nm) of visible light, and purple at the shorter end (380nm). This evolutionary process began more than 3 billion years ago with the formation of photopigments. These are molecules where light incident upon them will trigger a physical or chemical change. Photopigments capture photons which lead to the release of energy in the photopigment. This is may be used for photosynthesis or a behavioural reaction (a nerve reaction).

	\par
	Photocells contain a light sensitive patch of photopigments. Using a single cell we can capture light in 1 dimension and with multiple cells we can have better directon resolution.

	\section{Edge Detection}
	An intensity image is a data matrix whose values represent intensities within some range. Each element of the matrix corresponds to one image pixel.

	\section{Noise Filtering}

	\section{Advanced Edge Detection}

	\section{Hough Transform}

	\section{SIFT}

	\section{Face Recognition}

	\section{Motion}

	\section{ROC Analysis}

	\section{Object Recognition}

	\section{Model Based Object Recognition}
	\newpage
	\printindex

\end{document}
