\documentclass{article}
\usepackage{imakeidx}
\title{Computational Vision \linebreak Revision Notes}
\author{James Brown}
\makeindex
\begin{document}
	\pagenumbering{gobble}
	\maketitle
	\newpage
	\tableofcontents
	\newpage
	\pagenumbering{arabic}
	
	\section{Introduction}
	These are notes I have written in preparation of the 2017 Computation Vision exam. This year the module was run by Hamid Deghani (H.Dehghani@cs.bham.ac.uk).
	\linebreak \linebreak
	Computational vision is the acquisition of knowledge about objects and events in the environment through information processing of light emitted or reflected from objects. In short - we want to make a computer know what is where, by looking through information. We can also use computational vision to do automatic inference of properties of the world from images.
	
	\section{Human Vision}
	As humans we have evolved eyes which percieve the visible section of the electromagnetic spectrum, which falls between the wavelengths of 380nm - 760nm. Red light lies at the longer end (760nm) of visible light, and purple at the shorter end (380nm).
	
	\section{Edge Detection}
	
	\section{Noise Filtering}
	
	\section{Advanced Edge Detection}
	
	\section{Hough Transform}
	
	\section{SIFT}
	
	\section{Face Recognition}
	
	\section{Motion}
	
	\section{ROC Analysis}
	
	\section{Object Recognition}
	
	\section{Model Based Object Recognition}
	\newpage
	\printindex
	
\end{document}