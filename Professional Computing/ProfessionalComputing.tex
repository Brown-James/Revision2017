\documentclass{article}
\usepackage{imakeidx}
\usepackage{graphicx}
\graphicspath{{images/}}
\usepackage{mathtools}
\usepackage{geometry}
\geometry{a4paper,
total={170mm, 257mm},
left = 30mm,
right = 30mm,
bottom = 30mm,
top = 30mm
}
\title{Professional Computing \linebreak Revision Notes}
\author{James Brown}
\makeindex
\begin{document}
	\pagenumbering{gobble}
	\maketitle
	\newpage
	\tableofcontents
	\newpage
	\pagenumbering{arabic}
	
	\section{Introduction}
	These are notes I have written in preparation of the 2017 Professional Computing exam. This year the module was run by Iain Styles (I.B.Styles@cs.bham.ac.uk). I shall put an emphasis on topics I think are required for the essay question in the exam - 'Identify and discuss the ethical issues that the combination of big data analytics and video surveillance (including the use of drones) raises, and use your findings to critically analyse Mr Porter's position' - although I will cover all topics in at least some capacity for the multiple choice section.  As these notes a public, I must point out that \textbf{I am not a lawyer}. None of the information in these notes should be used as legal advice, please see a lawyer for that.
	
	\section{English Law}
	\subsection{What is Law?}
	Law is a complex and difficult entity and can be described as 'a set of laws which can be enforced by a court'. The law is more than just a simple set of rules - there are different systems of courts, different rules concerning how appeals are made and different rules which define how new laws work with old laws. Laws have \textbf{jurisdiction}\index{jurisdiction} - which is the geographical area which is governed by a single set of laws. For example, the Jurisdiction of English law is England, but the jurisdiction of California state law is solely California, and none of the other states within the USA. Jurisdiction is often not completely obvious in computer use. This means an offence involving a computer may involve laws applicable in other geographical areas than where you are currently sitting.
	\subsection{Criminal Law versus Civil Law}
	\textbf{Criminal law}\index{criminal law} is designed to protect society as a whole from wrongdoers. In all cases, the stance of 'innocent until proven guilty' is taken, and the offender must be proven \textbf{guilty beyond reasonable doubt}.	
	\par 
	\textbf{Civil law}\index{civil law} is for settling disputes between people (we can also count companies as people) and decisions are made based on the 'balance of probabilities'. Usually the objective of a civil law case is to obtain damages (money) or an injunction (court order). Litigation must be brought by one party of the dispute (the plaintiff\index{plaintiff}) against another (the defender\index{defender}). This module mostly concerns itself with civil law.
	
	\subsection{Torts}
	In common law, a tort\index{tort} is a civil wrong. The action may not necessarily be criminal or even illegal but has somehow caused harm. Torts are usually re-addressed through damages which are awarded. Possible causes of torts are negligence, nuisance or defamation (libel and slander). An example of a tort may be copy and pasting some broken code as a software engineer. This may not be against any contract signed, but would be negligence and break the duty of care, and as such can be considered a civil wrong.
	
	\section{The Computer Misuse Act}
	The \textbf{Computer Misuse Act}\index{computer misuse act} of 1990 covers three offences:
	\begin{itemize}
		\item Unauthorized access to a computer
		\item Unauthorized access to a computer to commit a serious crime
		\item Unauthorized modification of the contents of a computer
	\end{itemize}
	
	A person can be found as guilty of a crime violating the Computer Misuse act if either they or the computer in question is located in the UK at the time of the offence.
	
	\subsection{Section 1}
		A person is guilty of an offence is they cause a computer to perform any function with intent to secure access to any program or data held in any computer; the access they intend to secure is unauthorised; and they know at the time when they cause the computer to perform the function that this is the case. An offence committed is punishable by a fine of up to £5000 or 6 months imprisonment.
		\par
		Key points:
		\begin{itemize}
			\item Knowledge that you didn't have access and intent to secure access
			\item Just attempt is sufficient to be prosecuted
			\item There is no requirement for damage to be done
		\end{itemize}
	\subsection{Section 2}
	Section 2 covers unauthorized computer access to commit a more serious crime. For example, a blackmailer might hack into an email to gain evidence of an affair. It's not necessary for the crime to be carried out, intent to commit the crime just has to be shown. Punishment can be up to five years imprisonment or an unlimited fine.
	
	\subsection{Section 3}
	A person is also guilty of an offence if they do any act which causes an unauthorized modification of the contents of any computer; and at the time when they do the act they have the requisite intent and the requisite knowledge. Requisite intent covers:
	\begin{itemize}
		\item To impair the operation of any computer
		\item To prevent or hinder access to any program or data held in any computer
		\item To impair the operation of any such program or the reliability of any such data
	\end{itemize}
	
	The maximum penalty is five years imprisonment or an unlimited fine. Examples of offences that would oppose section three are spreading a virus, installing ransomware or even redirecting a browser homepage.
	
	\section{Data Protection Act}
	The UK introduced the Data Protection Act\index{data protection act} in 1984. It was designed to protect individuals against the use of inaccurate or incomplete personal information, the use of information by unauthorised persons and the use of information for reasons other that it was collected for. In 1998, the Data Protection Act underwent a major revision, which included a number of new definitions:
	\begin{itemize}
		\item \textbf{Data}\index{data}: information that is being processed automatically or is collected with that intention or recorded as part of a 'relevant filing system'.
		\item \textbf{Processing}\index{processing}: obtaining, recording or holding data or carrying out any operation on it.
		\item \textbf{Data Controller}\index{data controller}: Who controls why or how the data is processed.
		\item \textbf{Data Processor}\index{data processor}: Anybody who processes the data on behalf of the data controller.
		\item \textbf{Personal Data}\index{personal data}: Data which relates to a living person who can be identified using this data (possibly with other data the DC might have).
		\item \textbf{Sensitive Data}\index{sensitive data}: Personal data relating to racial, ethnic, religious, political, sexual (etc.) aspects of a person.
	\end{itemize}
	
	The revised data protection act was also comprised of many principles which must be abided by.
	
	\par 
	The \textbf{$1^{st}$ principle} states that data will be processed fairly and lawfully and in particular will not be processed unless (a) at least one condition in schedule 2 is met and (b) in the case of sensitive data at least one condition in schedule 3. Schedule 2 states that consent is given or there is some legal obligation to process the data (for example, tax returns or law enforcement). Schedule 3 states that \textit{explicit} consent is given.
	
	\par The \textbf{$2^{nd}$ principle} states that personal data shall be obtained only for one or more specified and lawful purposes, and shall not be further processed in any manner incompatible with that purpose or purposes. In short, data cannot be collected just in case it ends up being useful.	
	
	\par 
	The \textbf{$3^{rd}$ principle} states that personal data should be adequate, relevant and not excessive in relation to the purpose or purposes for which it is collected. This is often broken without thinking and by accident - an example includes asking for marital status when you are signing up to join a library.
	
	\par 
	The \textbf{$4^{th}$ principle} states that personal data should be accurate and where necessary kept up to date - this can be very hard to achieve in actual practice.
	
	\par 
	The \textbf{$5^{th}$ principle} states that personal data processed for any purpose or purposes should not be kept for longer than it is necessary for that purpose or those purposes. This poses a further question: how long is enough? 
	\begin{itemize}
		\item Financial data must be kept for up to 7 years for auditing purposes.
		\item Common advice is that emails should be kept for 7 years
		\item University exam results may be kept indefinitely
		\item CCTV data is routinely deleted after one month - this practice has implications for freedom of information requests. 
	\end{itemize}
	
	All procedures for data deletion must be rigorous and specified - this includes the deletion of backed up data.
	
	\par 
	The \textbf{$6^{th}$ principle} states that personal data should be processed in accordance with the rights of the data subjects under this act.
	
	\par 
	The \textbf{$7^{th}$ principle} states that appropriate technical and organisational measures shall be taken against unauthorised or unlawful processing of personal data and against accidental loss or destruction, or damage to personal data. In short - security is a legal requirement.
	
	\par 
	The \textbf{$8^{th}$ principle} states that personal data shall not be transferred to a country or territory outside the European Economic Area unless that country ensures an adequate level of protection for the rights and freedoms of data subjects in the relation of processing data. This specifically allows companies to transfer data over national boundaries.
	
	\section{Regulation of Investigatory Powers}
	The Regulation of Investigatory Powers Act\index{regulation of investigatory powers act} (2000) is a framework for lawful interception of computer, telephone and postal messages. It allows ISPs\index{internet service provider} and most employers to monitor communications without consent provided they are doing it to:
	\begin{itemize}
		\item Establish facts
		\item Ensure company regulations are being complied with
		\item Ascertain standards which ought to be achieved
		\item Prevent crime
		\item Investigate unauthorised use of telecommunications systems
		\item Ensure effective operation of systems (for example, detecting denial of service attacks)
		\item Find out whether a communication is business or personal
		\item Monitor but not record calls to confidential counselling helplines run free of charge by the business.
	\end{itemize}
	
	Any organisation monitoring communications without consent is required to make reasonable efforts to inform users that such interception might take place. RIPA also allows government agencies to ask for interception warrants to monitor communications to or from specific persons or organisations - for example the police, inland revenue or local councils. Councils are now limited to cases involving criminal offences that have at least a potential tariff of six months imprisonment.
		
	
	\section{Freedom of Information Act}
	The Freedom of Information act is an act to provide clear rights of access of information held by bodies in the public sector - with certain conditions and exceptions. If information is exempted from the Freedom of Information act then there is a duty on the public body to disclose where the public interest in disclosure outweighs the public interest in maintaining exemption. It is monitored by the Information Comissioner and "Information Tribunal" which have wide powers to enforce this act. It's a duty for all public bodies to adapt a scheme of publication of information which must be approved by the Information Comissioner. Under the Freedom of Information act information has a wide meaning - minutes of meetings are covered but personal data is not as that would violate the Data Protection Act.
	
	\par 
	Usually Freedom of Information requests must be answered within one month of reciept but sometimes this impossible. For some data this makes it effectively unaccessible - CCTV data for example. 
	
	\par Worth noting is that the US has an older and stronger Freedom of Information act which does include personal data. Any law enforcement agency must reveal all knowledge of criminal activities of a subject - the FBI claims to have handeled 300,000 of these requests.
	
	\section{Contracts, Consultancy and Liability}
	Contract law is old English law. It requires all parties to intend to make a contract and for all parties to be competent to make a contract. There also must be a 'consideration', meaning that each party much recieve something and provide something. Somewhat surprisingly, there is no need for either lawyers, writing or witnesses - but all of these things make the contract easier to enforce. These laws are typically fine for most things - except software.
	
	\par 
	Software projects are extremely high-risk - only 32\% of projects were completed on time, within budget and with the expected functionality in 2009 as reported by the Standish CHAOS Reports. They also reported that in 2009 24\% of projects failed.
	
	\par 
	Contracts are designed to protect both parties and there are three major kinds - fixed price contracts\index{fixed price contract}, time and materials\index{time and materials} and consultancy and contract hire.
	
	\par Fixed price contracts are typically for tailor made, bespoke systems. They typically have a short agreement of who the parties are, standard terms and conditions and a set of schedules/annexes. These consist of the particular requirements of the contract, what is supplied, any deadlines that may exist and what payments are to be made.
	
	\par A contract must specify what the 'product' to be produced is - the annex typically refers to a requirements specification\index{requirements specification}. It's common in software engineering for good requirement specifications to be difficult to achieve (and somewhat boring to produce too). Clients needs evolve over time and technologies change; the contract must address how these sorts of changes are accomodated. It must also have a method for calculating payment to deal with modifications.
	
	\par 
	The contract should also specify what is to be delivered. This is rarely just simply handing over the code as a text file. The contract may specify for source code, command line files for building executables, documentation, reference/training/operations manuals, training and test data with results. Ownership of the intellectual property rights must also be specified as well as confidentiality.
	
	\par 
	Upon completion of a contract, an invoice may be issued dependent on the structure of the contract. Payment is due within 30 days of issue of an invoice, and if payment is delayed by more than 30 days the company has the right to terminate the contract or apply a surcharge at an intrest rate of 2\% above the bank base lending rate. Such a clause is unlikely to ever be used as payment of a contract is much more likely to be staggered. Typically an initial payment of 15\% is made on signing of the contract, another 65\% is staged throughout the project, 25\% is made at acceptance of the software and there will be a final 10\% at the end of the contract. Staggered payments help to protect the supplier as the client may go out of business and also provide cash flow for the supplier.
	
	\par 
	Contracts may include penalty clauses - for example payment may be reduced by £5000 for each week the project overruns up to a maximum of £100,000 (10\%). Software is commonly delayed by penalty clauses are limited as suppliers are very reluctant to accept contracts which contain them. This leads to a smaller pool of reputable suppliers and usually increases the bid price by at least half the penalty. Should the software be really late and the penalty really high the supplier doesn't even have an incentive to complete the work.
	
	Contracts must provide a fixed set of acceptance tests (tasks, expected results, accuracy results etc.) as successful demonstration of the software constitutes acceptance. Sometimes tests are not 100\% successful. Because of this, warranty is provided. The standard amount of time for a warranty is 90 days during which any identified errors are fixed free of charge. Beyond the specified time, costs are subject to negoitiation. 
	
	\par 
	Some projects get cancelled, the client may go bust, may be merged with a larger company or the technology may become obsolete. The contract should detail what payments must be made to the supplier in the case of unfinished projects and what intellectual property rights exist.
	
	\par Contracts are very complex and litigation is very expensive. Contracts often may specify that in the case of a dispute the opinion of an independent arbitrator will decide the outcome. This avoids legal costs - the BCS maintains a list of qualified IT arbitrators.
	
	\par Time and materials\index{time and materials} contracts (also referred to as \textit{cost plus} contracts) are where a supplier agrees to develop software and recieves a payments based on costs incurred plus a daily rate. This may have a maximum price. These are usually cheaper than fixed price contracts and sometimes the projects is unclear making a fixed price contract impossible. There is currently a shift in IT away from time and materials contracts over to fixed price contracts - especially where public spending is involved.
	
	Lastly, contract hire\index{contract hire} and consultancy\index{consultancy} contracts offer a simpler alternative to complex fixed price contracts. In contract hire, the supplier will provide the services of their staff for a fixed period with agreed hourly/daily rates and the client then takes responsibility for managing the staff. Termination by either side of the contract may be done at short notice. In consultancy contracts, a report is usually produced by an expert doing analysis of a key part of the business. This is usually done at a fixed price but with small amounts of money involved.
	
	\par 
	Most suppliers are reluctant to agree to any liability for defective software or hardware. Standard terms and conditions usually limit liability to the project cost or even a fraction of that. The law disagrees with this stance under the Unfair Contract Act of 1977 which states that it is impossible to limit liability in the result of a death or personal injury. If the client is a consumer (a private person) and the supplier is acting as business, where the goods are of a type usual for private use, then the goods must be fit for purpose under the Sale of Goods Act (1979). Otherwise, under the Supply of Goods and Services Act (1982) then goods should be produced with 'reasonable' care. It's is however unclear as to whether software constitutes 'goods': if it's shrink wrapped licenced software then it probably is. Bespoke systems likely are not.
	
	\section{Intellectual Property, Copyrights, Patents, Trademarks and Confidential Information}
	
	Theft is the intentional taking of somebody else's property with the intention of permenantly depriving them of it. Laptops are tangible property\index{tangible property} and protected by laws related to theft and damage. On the other hand, information is intangible and governed by different laws. Some intangible property\index{intangible property} is very valuable - music, films, chemical formulae, software etc.
	
	\subsection{Copyright}
	\textbf{Copyright}\index{copyright} is governed by the Copyright, Design and Patents Act 1988, and Copyright (computer programs) Regulations 1992. The owner of any IP work has certain exclusive rights which are automatic and last for 70 years after the death of the author:
	
	\begin{itemize}
	\item To make copies of the work
	\item To issue copies of the work to the public (paid or free)
	\item To adapt the work (for example translations)
	\end{itemize}
	
	In order to make use of the item anybody must request permission, but this is often implicit (webpages for example). Copyright does not stop someone from publishing identical work, just work which has been exactly copied. It is not an infringement of the law to:
	
	\begin{itemize}
	\item Make a backup of a program you are authorized to use (only one copy allowed)
	\item Decompile code to correct any errors, or write code to interoperate with it.
	\item Sell your right to use the program (you must not keep a copy once sold).
	\end{itemize}
	
	These laws also extend to databases where content is of the authors own creation - in many situations this is not the case! Consider a fan database of every game played by Liverpool FC - this is not original content, just a lot of work. Databases are protected by copyright laws if there is substantial investment in obtaining, veryifying or presenting the contents of the database. This lasts for 15 years after the creation, and renews if the database is modified.
	
	Infringment of copyright laws may be a civil issue or a criminal issue. A primary infringement\index{copyright primary infringment} is a civil issue and is caused by infringing the rights of the copyright holder. A secondary infringement\index{copyright secondary infringment} is a crimnal issue and is a primary infringment in a business context, such as selling copies or using pirated software in a business. A lot of software uses some form of DRM, and providing information about how to avoid DRM is the same as actual copyright infringment.
	
	\subsubsection{Copyleft}
	Copyleft\index{copyleft} is a political/software philosiphy. Software is released as 'free' with no restrictions on re-use/modification/copying, except any resulting code must also be free. You can do anything with the code, except change the licence to not free. 
	
	\subsection{Patents}	
	Patents\index{patent} are governed by the Patents Act of 1977. Patents are a temporary right granted by the state enabling an inventor to prevent other people from copying their work without permission. Patents must be applied for but are far stronger legally than copyright. They are designed to encoruage innovation by rewarding the inventor a grace period to recoup development costs.
	
	\par 
	For a product to be patented it must be new, involve an inventive step (not an obvious solution that anyone well qualified might do), be capable of industrial application and not be in an area specifically excluded. These areas are:
	\begin{itemize}
	\item Scientific Theories
	\item Mathematical Methods
	\item Literary/dramatic/artistic work
	\item Presentation of information
	\item A scheme, rule, method for performing a mental act, playing a game or doing business, or \textbf{a program for a computer}
	\end{itemize}
	
	A patent is granted nationally, and there are trade schemes such as WIPO\index{WIPO} and European PO\index{european PO} but technically you still need to apply in each country. The date of initial application for the patent determines what is new, and patents often take 4 years to complete. As computing is a global business, the patent is required to be taken out in enough counties to ensure that it is challengable.
	
	\par 
	The US Patent and Trademark Office has always refused to patent software. This changed in 1981 when the accepted a patent for curing rubber which used software to determine the required heat. Currently, software may be patented if it is part of a product that be patented, controls some process with a physical effort or processes data that arises from the natural world. In Europe, by the European Patent Convention \& Copyright, Design and Patents Act 1988, software can technically cannot be patented but a few countries will do it anyway. Software patents are a \textbf{mess}.
	
	\subsection{Confidential Information}
	Often work involves an obligation of confidence. Software development for a company may reveal comercially sensitive information. There are usually coniditions in during the employment of employees to cover this. Even without such conditions, obligations may occur under equity. Obligation occurs if a reasonable person in such a position would reasonably understand that information was given in confidence. 
	
	\par 
	Under the Public Intrest Disclosure Act 1998, information can be disclosed if the employee believes any of the following might occur:
	\begin{itemize}
	\item A criminal offense
	\item Failure to comply with a legal obligation
	\item A miscarrage of justice
	\item Danger to health and safety
	\item Environmental damage
	\item Information that of of these has been concealed
	\end{itemize}
	
	\subsection{Trademarks and Passing Off}
	Any sign capable of being represented graphically which is capable of distinguishing goods and services may be trademarked\index{trademark}. Trademarks may or may not be registered, many countires require registration before there is any legal protection. In the UK this will be via the Patent Office. The US and Canada do not require registration before legal protection. 
	
	\par 
	The UK 1994 Act makes it illegal to apply an unauthorised trademark to goods, sell or hire goods with such a trademark or have in the course of business such goods. This is usually a criminal offense but civil actions may also be brought. UK law also protects unregistered trademarks by the Tort of Passing Off\index{passing off}. These are protected by civil actions, and are much less strong than registered trademarks.
	
	Patents\index{patent} are governed by the Patents Act of 1977. Patents are a temporary right granted by the state enabling an inventor to prevent other people from copying their work without permission. Patents must be applied for but are far stronger legally than copyright. They are designed to encoruage innovation by rewarding the inventor a grace period to recoup development costs.
	
	\par 
	For a product to be patented it must be new, involve an inventive step (not an obvious solution that anyone well qualified might do), be capable of industrial application and not be in an area specifically excluded. These areas are:
	\begin{itemize}
	\item Scientific Theories
	\item Mathematical Methods
	\item Literary/dramatic/artistic work
	\item Presentation of information
	\item A scheme, rule, method for performing a mental act, playing a game or doing business, or \textbf{a program for a computer}
	\end{itemize}
	
	A patent is granted nationally, and there are trade schemes such as WIPO\index{WIPO} and European PO\index{european PO} but technically you still need to apply in each country. The date of initial application for the patent determines what is new, and patents often take 4 years to complete. As computing is a global business, the patent is required to be taken out in enough counties to ensure that it is challengable.
	
	\par The US Patent and Trademark Office has always refused to patent software. This changed in 1981 when the accepted a patent for curing rubber which used software to determine the required heat. Currently, software may be patented if it is part of a product that be patented, controls some process with a physical effort or processes data that arises from the natural world. In Europe, by the European Patent Convention \& Copyright, Design and Patents Act 1988, software can technically cannot be patented but a few countries will do it anyway. Software patents are a \textbf{mess}.
	
	
	\section{Human Resources}

	\subsection{Recruitment and Selection}
	There are many different selection procedures that a company may follow. Commonly, one on one interviews with senior management are used. These are a reliable method of selection, especially if records can be reviewed for future applications. They also make it hard to comply with existing anti-discrimination legislation. 
	\par 
	Interviews may also be conducted by a panel. These may it very easy to comply with legislation, but research shows that they are not reliable - smooth talkers have a decided advantage. It's common to have independents present to avoid nepotism/corruption. There is evidence that panel interviews make bad decisions more often.
	
	\par 
	Tests may be issued, including ability tests (general ability in written skills etc), aptitude tests (ability to learn, programming ability etc.) or even personality tests. All of these types of tests can be practiced.		
		
	\par 
	References of a candidate will also be assessed. This is important for academic or public bodies, but less so in private companies. Often they are nothing more than a final check. There is a potential for civil action regarding references by the applicant if the references are unfair, and by the employer if references miss negative aspects. Recently references have become far less important.

	\subsection{Dismissal}
	Human Resources is responsible for ensuring that proper procedures are followed, and laws concerning dismissal are complex. The reason for dismissal must be valid, and correct procedure must be followed. Dismissal may be justified if done for the following reasons:
	\begin{itemize}
	\item Lack of capability
	\item Misconduct
	\item Statuatory duty/restriction prevents continued employment
	\item Redundancy
	\item Or any other reason
	\end{itemize}
	
	You cannot be dismissed for legal union activity, legal action to enforce employment rights or issues covered by anti-discrimination law (an example of this is maternity leave).
	
	\subsubsection{Statuatory Dismissal Procedure}
	From 2004-2009, employers were required to give reasons for dismissal in writing with a meeting arranged for all parties to give their case. If the employer goes ahead with the dismissal, the employee has the right to appeal (ideally to senior management). Unfiar dismissal could only be claimed after one year of employment, and any dismissal which breached the above was automatically considered unfair. 
	
	\par In 2009, this changed to the ACAS Code of Practice. This has no set procedure but a guide to best practice. 2 years are now required for any claim of unfair dismissal, with certian anti-discriminatory exemptions).
	\subsection{Redundancy}
	Redundancy is when a company no longer requires people to do a certain job (possibly because the company has gone bust). According to the law, employees are entitled to minimum levels of compensation based on age, salary and experience. Consultation is required if the company plans to make more than 20 employess redundant in 90 days or less. In the UK there are two concepts of redundancy. For the purposes or compensation, redundancy is where the employer no longer needs employees to do that job. For the purposes of consultation, redundancy is dismissal where reason is not related to individuals concerned. While difficult to select who to make redundant, Last In First Out is common.
	
	\subsection{Anti-Discrimination Laws}
	Form the 1970s a wide range of anti-discriminatory laws have been introduced. They provide similar (but not equal) protection from sexual discrimination, gender discrimination, discrimination against disability, religious discrimination and political discrimination. Each type of discrimination was monitored by a different government body. The Equality Act of 2010 consolidated all previous laws into a single act. It provides a single framework which guarantees anti-discrimination, equal pay and employment conditions and equal access to sales and services. It protects from discrimination about gender, sexual preference, marital status, age, race, religion, political views and disabilities. 
	
	\section{The Internet}
	
	\subsection{Defamation and Civil Law}
	Defamation covers statements which damage a persons reputation. In English law, \textbf{slander}\index{slander} regards spoken word and \textbf{libel}\index{libel} regards written word. The Defamation Act of 1996 defends anybody who is not the author, editor or publisher, or took reasonable care in relation to publication or did not know and had no reason to believe publication constituted a defamatory statement. ISPs typically receive a lot of such complaints. Assessing each of these is costly and time consuming - they are much more likely to just remove any potential defamation.
	
	\section{Ethics}
	
	\newpage
	\printindex	
	\end{document}
